\documentclass[a4paper]{article}
% \usepackage[margin=1.25in]{geometry}
\usepackage[inner=2.0cm,outer=2.0cm,top=2.5cm,bottom=2.5cm]{geometry}
% \usepackage{ctex}
\usepackage{color}
\usepackage{graphicx}
\usepackage{amssymb}
\usepackage{amsmath}
\usepackage{amsthm}
\usepackage{bm}
\usepackage{hyperref}
\usepackage{multirow}
\usepackage{mathtools}
\usepackage{enumerate}
\usepackage{tikz}
\usepackage{bm}
\usepackage{amssymb}
\usepackage{algorithm}
\usepackage{algorithmic}
\usepackage[colorlinks,linkcolor=blue,urlcolor=blue]{hyperref}

\renewcommand{\algorithmicrequire}{\textbf{Input:}} 
\renewcommand{\algorithmicensure}{\textbf{Output:}}


\newtheorem{definition}{定义}

\newcommand\bms{\bm{s}}
\newcommand\bmx{\bm{x}}
\newcommand\bmy{\bm{y}}
\newcommand\bmz{\bm{z}}

\newcommand\upx{\mathrm{x}}
\newcommand\upy{\mathrm{y}}
\newcommand{\pr}{\mathrm{P}}
\newcommand{\expect}[2][]{\mathrm{E}_{#1}[ #2 ]}
\newcommand{\fn}[1]{f^{\mathrm{n}}(\bms^{#1})}

\newcommand{\bsp}[1][n]{\{0,1\}^{#1}}
\newcommand\sps{\mathcal{S}}
\newcommand\spx{\mathcal{X}}
\newcommand\spy{\mathcal{Y}}
\newcommand*{\circled}[1]{\lower.7ex\hbox{\tikz\draw (0pt, 0pt)%
    circle (.5em) node {\makebox[1em][c]{\small #1}};}}
\usepackage{CJKutf8}

\newcommand{\homework}[5]{
    \pagestyle{myheadings}
    \thispagestyle{plain}
    \newpage
    \setcounter{page}{1}
    \noindent
    \begin{center}
    \framebox{
        \vbox{\vspace{2mm}
        \hbox to 6.28in { {\bf HSEA2022 \hfill #2} }
        \vspace{6mm}
        \hbox to 6.28in { {\Large \hfill #1 \hfill} }
        \vspace{6mm}
        \hbox to 6.28in { {\it Instructor: {\rm #3} \hfill Name: {\rm #4}, StudentId: {\rm #5}}}
        \vspace{2mm}}
    }
    \end{center}
    % \markboth{#4 -- #1}{#4 -- #1}
    \vspace*{4mm}
}
\begin{CJK}{UTF8}{gbsn}
\DeclareMathOperator*{\argmin}{argmin}
\def\R{\mathbb{R}}
\def\dom{\mathrm{dom}}


\newenvironment{solution}
{\color{blue} \paragraph{Solution.}}
{\newline \qed}




\begin{document}
%==========================Put your name and id here==========================
\homework{Homework 3}{Fall 2022}{Chao Qian}{Your name}{Your number}

\section{Problem 1: 求解LeadingOnes问题(20)}
请用适应层分析法来分析(1+1)-EA找到LeadingOnes问题最优解的期望运行时间上界。
\section{Problem 2: 求解OneMax问题(40)}
(1)~请用乘性漂移分析法求解(1+1)-EA算法找到OneMax问题最优解的期望运行时间上界(20)。
\\
\\
(2)~请用加性漂移分析法求解(1+1)-EA算法找到OneMax问题最优解的期望运行时间上界(20)。
\section{Problem 3: 求解COCZ问题(40)}
试求解GSEMO算法找到COCZ问题的帕累托前沿的期望运行时间上界。

\section{相关内容}

作业相关伪布尔函数问题的定义如下:
\begin{definition}[LeadingOnes]\label{def_leadingones}
	一个规模为$n$的LeadingOnes问题旨在找到一个$n$位的01串,以最大化
	\begin{equation}
		f(\bms)=\sum\limits^{n}_{i=1} \prod\limits^{i}_{j=1} s_j,
	\end{equation}
这里$s_j$指$\bms \in \{0,1\}^n$的第$j$位。
\end{definition}

\begin{definition}[OneMax]\label{def_onemax}
	一个规模为$n$的OneMax问题旨在找到一个$n$位的01串,以最大化
	\begin{equation}
		f(\bms)=\sum\limits^{n}_{i=1} s_i,
	\end{equation} 
	这里$s_i$指$\bms \in \{0,1\}^n$的第$i$位。
\end{definition}

\begin{definition}[COCZ]\label{def_bin}
	一个规模为$n$的COCZ: $\{0,1\}^n\rightarrow\mathbb{N}^2$问题旨在找到一个$n$位的01串,以最大化
	\begin{equation}
	COCZ(\bm{s})=\left(\sum\limits^{n}_{i=1}  s_i,\sum\limits^{n/2}_{i=1} s_i+\sum\limits^{n}_{i=n/2+1} (1-s_i)\right)
	\end{equation}
	这里$n$为偶数,且$s_i$指$\bms \in \{0,1\}^n$的第$i$位。
\end{definition}
对于二目标优化问题COCZ,可通过占优规则来比较两个解的优劣,具体定义如下:
\begin{definition}[占优规则]\label{def_domination}
    对于具有两目标$(f_1,f_2)$的解$\bms$和$\bm{s'}$来说,\\
1. 若~$\ \forall i: f_{i}(\bm{s})\ge f_{i}(\bm{s}')$,则$\bm{s}$~弱占优~$\bm{s}'$,即$\bm{s}$好于$\bm{s}'$,表示为$\bm{s}\succeq \bm{s}'$;\\
2. 若~$\bm{s}\succeq \bm{s}' \land \exists i: f_{i}(\bm{s})> f_{i}(\bm{s}')$,则$\bm{s}$~占优~$\bm{s}'$,即$\bm{s}$严格好于$\bm{s}'$,表示为$\bm{s}\succ \bm{s}'$;\\
3. 若既不满足$\bm{s}\succeq \bm{s}'$又不满足$\bm{s}'\succeq \bm{s}$,则$\bm{s}$~和~$\bm{s}'$~二者不可比.
\end{definition}
帕累托前沿的定义如下:
\begin{definition}[帕累托前沿]\label{pareto}
令$\mathcal{X}$代表问题的解空间。若解空间中不存在解能优于$\bms$,则称$\bms$为帕累托最优解。所有帕累托最优解的目标向量集合称为帕累托前沿。
\end{definition}
(1+1)-EA和GSEMO算法的基本流程如算法~\ref{algo:(1+1)-EA}和算法~\ref{algo:GSEMO}所示:
\begin{algorithm}[htb]
	\begin{algorithmic}[1]
		\REQUIRE{伪布尔函数$f:\bsp\rightarrow \mathbb{R}$}
		\ENSURE{$\bsp$中的一个解}
		\STATE 随机均匀地从$\{0,1\}^n$中选择一个解$\bms$作为初始解
		\WHILE{算法终止条件不满足}
		\STATE $\bms'\gets$将$\bms$的每一位独立地以$1/n$的概率翻转
		\IF{$f(\bms')\ge f(\bms)$}
		\STATE $\bms\gets\bms'$
		\ENDIF
		\ENDWHILE
		\RETURN{$\bms$}
		\caption{(1+1)-EA}
		\label{algo:(1+1)-EA}
	\end{algorithmic}
\end{algorithm}
\begin{algorithm}[htb]
	\begin{algorithmic}[1]
		\STATE 随机均匀地从$\{0,1\}^n$中选择一个解$\bms$作为初始解
		\STATE 将初始解放入种群$P\leftarrow\{\bms\}$
		\WHILE{算法终止条件不满足}
		\STATE 随机均匀地从种群$P$中挑选出解$\bms$
		\STATE $\bms'\gets$将$\bms$的每一位独立地以$1/n$的概率翻转
		\IF {$\nexists \bm{z} \in P$ 使得$\bm{z} \succ \bm{s}'$}
\STATE $P = (P - \{\bm{z} \in P \mid \bm{s}' \succeq \bm z\}) \cup \{\bm{s}'\}$
\ENDIF
		\ENDWHILE
		
		\caption{GSEMO}
		\label{algo:GSEMO}
	\end{algorithmic}
\end{algorithm}
\section{提交与评分}
提交一份pdf文档,并发送到\href{mailto:liudx@lamda.nju.edu.cn}{liudx@lamda.nju.edu.cn},\textbf{12月17日23:59}截止。延期提交的折扣为-10/天,即每延迟一天,本次作业得分减10。请合理分配时间。
\begin{itemize}
	\item Pdf文档命名方式:“学号-姓名.pdf”,例如“MG1937000-张三.pdf”;
	\item 邮件标题命名:“HSEA第三次作业-学号-姓名”,例如“HSEA第三次作业-MG1937000-张三”。
\end{itemize}
注意,pdf可以用latex/word/markdown等方式生成,但是不要用手写证明的照片。


作业的评分主要参考以下几点:
\begin{enumerate}
	\item 结论的紧致性。
	\item 证明过程的完整性以及正确性。例如在使用分析工具时是否充分考虑了工具的条件,公式推导是否完整、以及是否有错误。
	\item 文档的细节。例如是否出现符号错误,文档格式是否混乱。
\end{enumerate}

\textbf{若发现作业出现雷同的情况,会根据相关规定给予惩罚,详情请参考课程主页中“学术诚信”的相关内容。}请同学们务必独立完成作业!


\end{document}
\end{CJK}