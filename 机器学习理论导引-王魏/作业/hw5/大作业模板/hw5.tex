\documentclass[a4paper,UTF8]{article}
\usepackage{ctex}
\usepackage[margin=1.25in]{geometry}
\usepackage{color}
\usepackage{graphicx}
\usepackage{amssymb}
\usepackage{amsmath}
\usepackage{amsthm}
\usepackage{enumerate}
\usepackage{bm}
\usepackage{hyperref}
\usepackage{pgfplots}
\usepackage{epsfig}
\usepackage{color}
\usepackage{tcolorbox}
\usepackage{mdframed}
\usepackage{lipsum}


\usepackage{natbib}
\newmdtheoremenv{thm-box}{myThm}
\newmdtheoremenv{prop-box}{Proposition}
\newmdtheoremenv{def-box}{定义}

\setlength{\evensidemargin}{.25in}
\setlength{\textwidth}{6in}
\setlength{\topmargin}{-0.5in}
\setlength{\topmargin}{-0.5in}
% \setlength{\textheight}{9.5in}
%%%%%%%%%%%%%%%%%%此处用于设置页眉页脚%%%%%%%%%%%%%%%%%%
\usepackage{fancyhdr}                                
\usepackage{lastpage}                                           
\usepackage{layout}                                             
\footskip = 12pt 
\pagestyle{fancy}                    % 设置页眉                 
\lhead{2023年春季}                    
\chead{机器学习理论研究导引}                                                
% \rhead{第\thepage/\pageref{LastPage}页} 
\rhead{大作业}                     
\cfoot{\thepage}                                                
\renewcommand{\headrulewidth}{1pt}  			%页眉线宽,设为0可以去页眉线
\setlength{\skip\footins}{0.5cm}    			%脚注与正文的距离           
\renewcommand{\footrulewidth}{0pt}  			%页脚线宽,设为0可以去页脚线
\pgfplotsset{compat=1.16}

\makeatletter 									%设置双线页眉                                        
\def\headrule{{\if@fancyplain\let\headrulewidth\plainheadrulewidth\fi%
\hrule\@height 1.0pt \@width\headwidth\vskip1pt	%上面线为1pt粗  
\hrule\@height 0.5pt\@width\headwidth  			%下面0.5pt粗            
\vskip-2\headrulewidth\vskip-1pt}      			%两条线的距离1pt        
 \vspace{6mm}}     								%双线与下面正文之间的垂直间距              
\makeatother  

%%%%%%%%%%%%%%%%%%%%%%%%%%%%%%%%%%%%%%%%%%%%%%
\numberwithin{equation}{section}
%\usepackage[thmmarks, amsmath, thref]{ntheorem}
\newtheorem{myThm}{myThm}
\newtheorem*{myDef}{Definition}
\newtheorem*{mySol}{Solution}
\newtheorem*{myProof}{Proof}
\newtheorem*{myRemark}{备注}
\renewcommand{\tilde}{\widetilde}
\renewcommand{\hat}{\widehat}
\newcommand{\indep}{\rotatebox[origin=c]{90}{$\models$}}
\newcommand*\diff{\mathop{}\!\mathrm{d}}

\usepackage{multirow}

%--

%--
\begin{document}
\title{机器学习理论研究导引\\
大作业}
\author{你的姓名\, 你的学号\, 你的院系} 
\date{}
\maketitle
%%%%%%%% 注意: 使用XeLatex 编译可能会报错,请使用 pdfLaTex 编译 %%%%%%%

\section*{作业提交注意事项}
\begin{tcolorbox}
  \begin{enumerate}
      \item[(1)] 本次作业提交截止时间为~\textcolor{red}{\textbf{2023/07/02  23:59:59}}, 截止时间后不再接收作业, 本次作业记零分; 
      \item[(2)] 作业提交方式: 使用此LaTex模板书写解答, 只需提交编译生成的pdf文件, 将pdf文件提交至南大网盘:
      \newline https://box.nju.edu.cn/u/d/c1944351a2de4e93ada5/
      \item[(3)] pdf 文件命名方式:学号-姓名-作业号-v版本号, 例~ MG1900000-张三-5-v1;如果需要更改已提交的解答,请在截止时间之前提交新版本的解答,并将版本号加一;
      \item[(4)] 未按照要求提交作业,或~pdf~命名方式不正确,将会被扣除部分作业分数. 
  \end{enumerate}
\end{tcolorbox}

\begin{abstract}

此处为文章摘要

\end{abstract}


\newpage
\section{此处为第一章标题}
使用\textbackslash{}citep 命令引用参考文献,  例如\citep{book/mohri2018foundations} \citep{journal/Vapnik1971} \citep{conf/Sugiyama2006local}

\section{此处为第二章标题}

\section{可自由添加后续章节}

%% 参考文献列表  参考文献位于 ref.bib 中, 文件中给出了书籍 期刊 会议的格式, 请参照对应格式添加参考文献
\bibliographystyle{abbrvnat}
\bibliography{ref}
\end{document}