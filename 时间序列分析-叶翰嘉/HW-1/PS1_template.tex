\documentclass[a4paper,UTF8]{article}
\usepackage{ctex}
\usepackage[margin=1.25in]{geometry}
\usepackage{color}
\usepackage{graphicx}
\usepackage{amssymb}
\usepackage{amsmath}
\usepackage{amsthm}
\usepackage{enumerate}
\usepackage{bm}
\usepackage{hyperref}
\usepackage{epsfig}
\usepackage{color}
\usepackage{booktabs}
\usepackage{tcolorbox}
\usepackage{mdframed}
\usepackage{lipsum}
\newmdtheoremenv{thm-box}{myThm}
\newmdtheoremenv{prop-box}{Proposition}
\newmdtheoremenv{def-box}{定义}
\usepackage{hyperref}
\setlength{\evensidemargin}{.25in}
\setlength{\textwidth}{6in}
\setlength{\topmargin}{-0.5in}
\setlength{\topmargin}{-0.5in}
% \setlength{\textheight}{9.5in}
%%%%%%%%%%%%%%%%%%此处用于设置页眉页脚%%%%%%%%%%%%%%%%%%
\usepackage{fancyhdr}
\usepackage{lastpage}
\usepackage{layout}
\footskip = 10pt
\pagestyle{fancy}                    % 设置页眉
\lhead{2022年秋季}
\chead{时间序列分析}
% \rhead{第\thepage/\pageref{LastPage}页}
\rhead{作业一}
\cfoot{\thepage}
\renewcommand{\headrulewidth}{1pt}  			%页眉线宽,设为0可以去页眉线
\setlength{\skip\footins}{0.5cm}    			%脚注与正文的距离
\renewcommand{\footrulewidth}{0pt}  			%页脚线宽,设为0可以去页脚线

\makeatletter 									%设置双线页眉
\def\headrule{{\if@fancyplain\let\headrulewidth\plainheadrulewidth\fi%
		\hrule\@height 1.0pt \@width\headwidth\vskip1pt	%上面线为1pt粗
		\hrule\@height 0.5pt\@width\headwidth  			%下面0.5pt粗
		\vskip-2\headrulewidth\vskip-1pt}      			%两条线的距离1pt
	\vspace{6mm}}     								%双线与下面正文之间的垂直间距
\makeatother

%%%%%%%%%%%%%%%%%%%%%%%%%%%%%%%%%%%%%%%%%%%%%%
\numberwithin{equation}{section}
%\usepackage[thmmarks, amsmath, thref]{ntheorem}
\newtheorem{myThm}{myThm}
\newtheorem*{myDef}{Definition}
\newtheorem*{mySol}{Solution}
\newtheorem*{myProof}{Proof}
\newcommand{\indep}{\rotatebox[origin=c]{90}{$\models$}}
\newcommand*\diff{\mathop{}\!\mathrm{d}}

\usepackage{multirow}
\usepackage{subfigure}
\graphicspath{{code/}}

%--

%--
\begin{document}
	\title{时间序列分析\\
		作业一}
	\author{502022370071, 庄镇华, \href{mailto:zhuangzh@lamda.nju.edu.cn}{zhuangzh@lamda.nju.edu.cn}}
	\maketitle

	\section*{作业提交注意事项}
	\begin{tcolorbox}
		\begin{enumerate}
			\item[(1)] 请严格参照教学立方网站所述提交作业,压缩包命名统一为{\color{red}学号\_姓名.zip};
			\item[(2)] 未按照要求提交作业,或提交作业格式不正确,将会被扣除部分作业分数;
			\item[(3)] 除非有特殊情况(如因病缓交),否则截止时间后不接收作业,本次作业记零分。
		\end{enumerate}
	\end{tcolorbox}

	\section{[100pts] 预处理、简单模型和评价指标}
	\href{https://github.com/zhouhaoyi/ETDataset}{ETT (Electricity Transformer Temperature)数据集}是一个多变量时间序列数据集,本次实验选取ETTh1中"油温(OT)"组分作为单独的单变量时间序列,该序列记录了每个小时(h)的油温变化,学生需要以此序列为实验对象,实现若干预处理方法、简单模型和评价指标。数据集、代码分别保存在data、code文件夹下。学生需要完成的任务如下:
	\begin{enumerate}[ {(}1{)}]
		\item 在transforms.py中实现归一化(normalization), 标准化(Standardization),平均归一化(Mean Normalization),Box-Cox变换,需继承Transform类并实现其抽象方法;
		\item 在models.py中实现$Naive_1$,$Naive_s$(以24h为周期),Drift模型,需继承ForecastModel类并实现其抽象方法;
		\item 在metrics.py中实现MSE,MAE,MAPE,sMAPE,	MASE。
		\item 修改并运行main.py,汇报(2)中不同方法,在(1)中不同变换下,用(3)中不同指标衡量的性能,以表格形式呈现,表格示例如表\ref{tb:example}所示。绘制并报告(2)中模型真实序列与预测序列的曲线图(变换方式任选,但需在解答中说明)。
	\end{enumerate}
	注:utils.py中包含了一些有用的函数,请勿对其进行修改,如有疑问可联系助教。最终需提交的文件为: 1. 修改后的代码,\underline{要求附加一个markdown格式的文件README.md},说明如何复现报告中的结果。2. pdf形式的报告,报告需\textit{描述各个功能的实现(例如以数学公式的形式)}并报告结果,写于\textbf{solution}部分即可。
	\begin{mySol}
		此处用于报告(中英文均可)
		\begin{table}[]
			\centering
			\caption{表格示例}
			\begin{tabular}{ccccc}
				\toprule
				Model                      & Transform & MAE & MSE & MAPE \\
				\midrule
				\multirow{3}{*}{Drift}     & None      &     &     &      \\
				& Normalize &     &     &      \\
				& Box-Cox   &     &     &      \\
				\midrule
				\multirow{3}{*}{$Naive_1$} & None      &     &     &      \\
				& Normalize &     &     &      \\
				& Box-Cox   &     &     &    \\
				\bottomrule
			\end{tabular}
			\label{tb:example}
		\end{table}

		\begin{table}[]
			\centering
			\caption{不同方法在不同变换下的性能指标}
			\begin{tabular}{ccccccc}
				\toprule
				Model                      & Transform & MSE & MAE & MAPE & sMAPE & MASE \\
				\midrule
				\multirow{8}{*}{$Naive_1$} & None & 18.65752471 & 3.535363442 &  111.7803963 & 53.57583421 &  2.16576395 \\
				& Normalize  & 18.65752471 & 3.535363442 &  111.7803963 & 53.57583421 &  2.16576395 \\
				& Standardization  & 18.65752471 & 3.535363442 &  111.7803963 & 53.57583421 &  2.16576395 \\
				& MeanNormalize  & 18.65752471 & 3.535363442 &  111.7803963 & 53.57583421 &  2.16576395 \\
				& Box-Cox(-1)    & 18.65752471 & 3.535363442 &  111.7803963 & 53.57583421 &  2.16576395 \\
				& Box-Cox(0)    & 18.65752471 & 3.535363442 &  111.7803963 & 53.57583421 &  2.16576395 \\
				& Box-Cox(0.5)    & 18.65752471 & 3.535363442 &  111.7803963 & 53.57583421 &  2.16576395 \\
				& Box-Cox(1)    & 18.65752471 & 3.535363442 &  111.7803963 & 53.57583421 &  2.16576395 \\
				\midrule
				\multirow{8}{*}{$Naive_s$} & None & 28.80465239 & 4.399937512 & 138.6976645 & 59.92193047 & 2.6954021\\
				& Normalize   & 28.80465239 & 4.399937512 & 138.6976645 & 59.92193047 & 2.6954021\\
				& Standardization  & 28.80465239 & 4.399937512 & 138.6976645 & 59.92193047 & 2.6954021\\
				& MeanNormalize   & 28.80465239 & 4.399937512 & 138.6976645 & 59.92193047 & 2.6954021\\
				& Box-Cox(-1)     & 28.80465239 & 4.399937512 & 138.6976645 & 59.92193047 & 2.6954021\\
				& Box-Cox(0)     & 28.80465239 & 4.399937512 & 138.6976645 & 59.92193047 & 2.6954021\\
				& Box-Cox(0.5)     & 28.80465239 & 4.399937512 & 138.6976645 & 59.92193047 & 2.6954021\\
				& Box-Cox(1)     & 28.80465239 & 4.399937512 & 138.6976645 & 59.92193047  & 2.6954021\\
				\midrule
				\multirow{8}{*}{Drift} & None    & 42.38380405
				    &  5.030783023
				       &  85.42761343
				         & 99.99108322
				           & 3.081858115 \\
				& Normalize & 42.38380405
				&  5.030783023
				&  85.42761343
				& 99.99108322
				& 3.081858115 \\
				& Standardization  & 42.38380405
				&  5.030783023
				&  85.42761343
				& 99.99108322
				& 3.081858115 \\
				& MeanNormalize & 42.38380405
				&  5.030783023
				&  85.42761343
				& 99.99108322
				& 3.081858115 \\
				& Box-Cox(-1) & 16.81469076	& 3.503086382	& 93.86983314 &	55.78907566	& 2.145991021 \\
				& Box-Cox(0) &	19.77990009 & 	3.735680781	& 87.22352571	& 61.86052273	& 2.28847837 \\
				& Box-Cox(0.5) & 25.27639353 &	4.090526961	& 84.05539909	& 72.02739739	& 2.50585717 \\
				& Box-Cox(1) & 42.38380405	& 5.030783023	& 85.42761343	& 99.99108322 &	3.081858115 \\
				\bottomrule
			\end{tabular}
			\label{tb:res}
		\end{table}

		% 解答部分
		\begin{enumerate}[ {(}1{)}]
			\item 数据变换即对数据进行规范化处理,以便于后续的信息挖掘。常见的数据变换包括:归一化(normalization), 标准化(Standardization),平均归一化(Mean Normalization),Box-Cox变换等。下面以数学公式的形式描述各种变换的实现方法。\\
			归一化(normalization):将时序数据取值限制在$[0,1]$\\
			$$y_t'=\frac{y_t-y_{min}}{y_{max}-y_{min}}$$
			标准化(Standardization)/ Z-Score:将时序数据变换为0均值以及标准方差\\
			$$y_t'=\frac{y_t-\mu}{\sigma}$$
			平均归一化(Mean Normalization):\\
			$$y_t'=\frac{y_t-\mu}{y_{max}-y_{min}}$$
			Box-Cox变换用于分布“正态”程度矫正:其中$\lambda$为超参数,实验选取了-1,0,0.5,1四个值\\
			$$ y_t^{(\lambda)} =\left\{
			\begin{array}{lcl}
			\frac{(y_t^\lambda-1)}{\lambda}       &      & {\lambda \neq 0}\\
			\log y_t     &      & {\lambda = 0}
			\end{array} \right. $$
			由于实验数据集含有负数,因此使用二参数Box-Cox变换,由于最小值大于-5,因此选取$\lambda_2=5$\\
			$$ y_t^{(\lambda)} =\left\{
			\begin{array}{lcl}
			\frac{((y_t+\lambda_2)^{\lambda_1}-1)}{\lambda_1}       &      & {\lambda_1 \neq 0}\\
			\log (y_t + \lambda_2)     &      & {\lambda_1 = 0}
			\end{array} \right. $$
			\item 在给定时间序列\{$y_1$, $y_2$, … , $y_n$\}的情况下,以数学公式的形式描述各种模型的实现方法。\\
			$Naive_1$:常数预测模型。使用最后一个观察值对后续样本进行预测\\
			$$ y_{n+h}' = y_n $$
			$Naive_s$:对于周期性时间序列,使用上一个周期同期的观察值作为当前时刻的预测值,实验周期$m=24$。
			$$ y_{n+h}' = y_{n+h-m(\lfloor \frac{h-1}{m} \rfloor + 1)} $$
			Drift方法:充分考虑到时间序列前后的变化。每两个相邻的时间序列可以计算变化值的均值,用这一变化值指导后续的预测。
			$$ y_{n+h}' = y_n + \frac{h}{n - 1}\sum_{t=2}^n(y_t - y_{t - 1})=y_n+h(\frac{y_n-y_1}{n -1}) $$
			\item 下面以数学公式的形式描述各种指标的计算方法。\\
			MSE (Mean Square Error)\\
			$$ MSE = \frac{1}{H}\sum_{i=1}^{H}(y_{n+i}-y_{n+i}')^2$$
			MAE (Mean Absolute Error)\\
			$$ MAE = \frac{1}{H}\sum_{i=1}^{H}|y_{n+i}-y_{n+i}'|$$
			MAPE (Mean Absolute Percentage Error)\\
			$$ MAPE = \frac{100}{H}\sum_{i=1}^{H}\frac{|y_{n+i}-y_{n+i}'|}{|y_{n+i}|}$$
			sMAPE (symmetric MAPE)\\
			$$ sMAPE = \frac{200}{H}\sum_{i=1}^{H}\frac{|y_{n+i}-y_{n+i}'|}{|y_{n+i}|+|y_{n+i}'|}$$
			MASE (Mean Absolute Scaled Error)
			$$ MASE = \frac{1}{H}\sum_{i=1}^{H}\frac{|y_{n+i}-y_{n+i}'|}{\frac{1}{n+H-m}\sum_{j=m+1}^{n+H}|y_j-y_{j-m}|}$$
			\item
			不同方法在不同变换下的性能指标如表\ref{tb:res}所示,针对二参数Box-Cox模型,我们选取$\lambda_1=-1,0,0.5,1$观察实验结果,$\lambda_2=5$为定值用于调整负值;\\
			三种模型预测序列与真实序列的曲线如图\ref{fig:1}所示,选取两种变换方式,分别为Standardization和Box-Cox(0)变换。\\
		
		\end{enumerate} 



		~\\
		~\\
		~\\
	\end{mySol}
\begin{figure}
	\caption{三种模型预测序列与真实序列的曲线}
	\begin{minipage}{0.48\linewidth}
		\centerline{\includegraphics[scale=0.55]{Naive1_Standardization.png}}
		\centerline{$Naive_1$模型Standardization变换}
	\end{minipage}
	\hfill 
	\begin{minipage}{0.48\linewidth}
		\centerline{\includegraphics[scale=0.55]{Naive1_BoxCox-0.png}}
		\centerline{$Naive_1$模型Box-Cox(0)变换}
	\end{minipage}
	\vfill
	\begin{minipage}{0.48\linewidth}
		\centerline{\includegraphics[scale=0.55]{NaiveS_Standardization.png}}
		\centerline{$Naive_s$模型Standardization变换}
	\end{minipage}
	\hfill 
	\begin{minipage}{0.48\linewidth}
		\centerline{\includegraphics[scale=0.55]{NaiveS_BoxCox-0.png}}
		\centerline{$Naive_s$模型Box-Cox(0)变换}
	\end{minipage}
	\vfill
	\begin{minipage}{0.48\linewidth}
		\centerline{\includegraphics[scale=0.55]{Drift_Standardization.png}}
		\centerline{$Drift$模型Standardization变换}
	\end{minipage}
	\hfill 
	\begin{minipage}{0.48\linewidth}
		\centerline{\includegraphics[scale=0.55]{Drift_BoxCox-0.png}}
		\centerline{$Drift$模型Box-Cox(0)变换}
	\end{minipage}
	\label{fig:1}
	
\end{figure}
	\newpage
	\section{[附加题20pts] 周期的影响}
	在上一题中,我们指定了$Naive_s$方法中的周期为天(24h),本题中学生需要尝试使用不同的周期,考察不同周期下$Naive_s$模型的性能变化。请绘制出模型性能随不同周期的变化曲线。在ETTh1的OT序列上,最好的周期是什么?你可以得出什么结论?
	\begin{mySol}
		此处用于报告(中英文均可)
		~\\分别计算周期$T = 1h, 2h, 3h, 4h , 5h, 6h, 9h, 12h(0.5d), 24h(1d), 36h, 48h(2d), 60h, 72h(3d)$, $120h$\\$(5d)$, $168h(7d)$,$336h(14d), 720h(30d), 1440h(60d), 2160h(90d)$的模型性能,其中变换方式取Standardization,得到结果如表\ref{tb:res2}所示,将其可视化为图\ref{fig:2},可以得知最好的周期是 $3$ 小时。
		~\\可以得出结论:对于$Naive_s$模型,周期对其性能有很大的影响,过短的周期无法把握整体的趋势,过长的周期无法刻画瞬时的变化,只有选取恰当的周期才能达到最好的效果。
		~\\
		\\\\
				\begin{table}[]
			\centering
			\caption{不同周期对$Naive_s$模型性能的影响}
			\begin{tabular}{ccccccc}
				\toprule
				T/h & MSE & MAE & MAPE & sMAPE & MASE \\
				\midrule
				1	& 18.65752471	& 3.535363442	& 111.7803963	& 53.57583421	& 2.16576395 \\
				2	& 18.34002584	& 3.509119488	& 110.6913492	& 53.37720748	& 2.149686902 \\
				3	& 18.02599282	& 3.483062658	& 109.5532941	& 53.18038683	& 2.133724485 \\
				4	& 19.14936796	& 3.57654066	& 113.0188586	& 53.88935066	& 2.190989118 \\
			    5	& 20.5444861	& 3.698082518	& 117.2391948	& 54.81394785	& 2.26544567 \\
				6	& 21.89197442	& 3.808982693	& 120.693299	& 55.62994634	& 2.333383127 \\
				9	& 26.25339061	& 4.192032928	& 132.3722014	& 58.4604395	& 2.568039734 \\
				12	& 28.08862302	& 4.350078946	& 136.9206051	& 59.60811817	& 2.664858738 \\
				24	& 28.80465239	& 4.399937512	& 138.6976645	& 59.92193047	& 2.6954021 \\
				36	& 33.99119983	& 4.838767039	& 149.8191526	& 62.98541309	& 2.964229106 \\
				48	& 30.40258072	& 4.52262551	& 140.6586855	& 60.69909803	& 2.770560778 \\
				60	& 31.6850802	& 4.640361143	& 145.223024	& 61.61661936	& 2.842685637 \\
				72	& 30.35224837	& 4.513158591	& 141.5767519	& 60.65336168	& 2.764761343 \\
				120	& 26.97899457	& 4.184694546	& 132.6652017	& 58.10934586	& 2.563544241 \\
				168	& 24.06728953	& 4.024854047	& 122.7379699	& 57.40610562	& 2.465625937 \\
				336	& 22.83873419	& 3.907338463	& 110.5461512	& 57.45510201	& 2.39363588 \\
				720	& 34.95754482	& 4.781684738	& 141.498658	& 62.21897926	& 2.929260484 \\
				1440	& 51.54356031	& 5.953657632	& 171.4524685	& 68.87921892	& 3.64721118 \\
				2160	& 83.71203853	& 7.778575769	& 216.2894885	& 78.15369892	& 4.765156188 \\
				
				\bottomrule
			\end{tabular}
			\label{tb:res2}
		\end{table}
	
		\begin{figure}
			\caption{不同周期对$Naive_s$模型性能的影响}
			\begin{minipage}{0.5\linewidth}
				\centerline{\includegraphics[scale=0.40]{mse_period.png}}
				\centerline{MSE}
			\end{minipage}
			\hfill 
			\begin{minipage}{0.5\linewidth}
				\centerline{\includegraphics[scale=0.40]{mae_period.png}}
				\centerline{MAE}
			\end{minipage}
			\vfill
			\begin{minipage}{0.5\linewidth}
				\centerline{\includegraphics[scale=0.39]{mape\%_period.png}}
				\centerline{MAPE}
			\end{minipage}
			\hfill 
			\begin{minipage}{0.5\linewidth}
				\centerline{\includegraphics[scale=0.39]{smape\%_period.png}}
				\centerline{sMAPE}
			\end{minipage}
			\vfill
			\begin{minipage}{1.0\linewidth}
				\centerline{\includegraphics[scale=0.40]{mase_period.png}}
				\centerline{MASE}
			\end{minipage}
			\label{fig:2}
		\end{figure}
	\end{mySol}


\end{document}